\documentclass[10pt,a4paper,twoside]{article}
\usepackage[dutch]{babel}
\usepackage{graphicx}
\usepackage{float,flafter}
\usepackage{hyperref}
\usepackage{inputenc}
%zet de bladspiegel :
\setlength\paperwidth{20.999cm}\setlength\paperheight{29.699cm}\setlength\voffset{-1in}\setlength\hoffset{-1in}\setlength\topmargin{1.499cm}\setlength\headheight{12pt}\setlength\headsep{0cm}\setlength\footskip{1.131cm}\setlength\textheight{25cm}\setlength\oddsidemargin{2.499cm}\setlength\textwidth{15.999cm}

\begin{document}
\begin{center}
\hrule

\vspace{.3cm}
{\bf {\Large Assignment 3 }}\\
{\bf {\huge Moravec's Paradox}}
\vspace{.2cm}
\end{center}
{\bf Name:}  Priyanshu Gupta\\
{\bf Roll no:}  19111042 \\
{\bf Branch: }  Biomedical Engineering \hspace{\fill}  22 July, 2021 \\
\hrule

\vspace{.4cm}
\begin{center}
{\textbf{\large A one page summary of "Moravec's Paradox".}} \\
\end{center}
\textbf{Moravec's paradox} is the observation by artificial intelligence and robotics researchers that, contrary to traditional assumptions, reasoning requires very little computation, but sensorimotor skills require enormous computational resources. \\
It means that unconscious tasks simple for humans like recognition and smelling demand more data and intelligence from a computer while conscious tasks like chess is just a algorithms. 
 

\section*{The biological basis of human skills}
One possible explanation of the paradox is based on \textbf{evolution}. The process of natural selection tend to preserve design improvements and optimizations. The older a skill is, the more time natural selection has had to improve the design. \\
According to \textbf{Moravec}, the oldest human skills or unconscious tasks like recognizing a face, judging people's motivations, paying attention to things that are interesting, and abstract thoughts are a result of skills reinforced through millions of years of evolution. While skills that have appeared more recently like engineering, games, logic and scientific reasoning are hard for us because these are skills and techniques that were acquired recently.\\
Therefore, we should expect skills that appear effortless to be difficult to reverse-engineering.


\section*{Historical influence on artificial intelligence}
In the early days of artificial intelligence research, logic and algebra(hard problems) difficult for people, were solved successfully by writing programs that used logic. Solving those they thought they would be able to create thinking machines in just a few decades but they were wrong because the "easy" problems of vision and commonsense reasoning were not easy at all, but incredibly difficult.\\
Earlier, intelligence was best characterized as the things that highly educated male scientists found challenging such as solving complicated word algebra problems. So, a scientist named Brooks decided to build intelligent machines that had "No cognition”, just sensing and action. This new direction he called \textbf{Nouvelle AI"}was highly influential on robotics research and AI. 



\end{document}